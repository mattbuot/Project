%
% File acl2014.tex
%
% Contact: giovanni.colavizza@epfl.ch
%%
%% Based on the style files for ACL-2013, which were, in turn,
%% Based on the style files for ACL-2012, which were, in turn,
%% based on the style files for ACL-2011, which were, in turn, 
%% based on the style files for ACL-2010, which were, in turn, 
%% based on the style files for ACL-IJCNLP-2009, which were, in turn,
%% based on the style files for EACL-2009 and IJCNLP-2008...

%% Based on the style files for EACL 2006 by 
%%e.agirre@ehu.es or Sergi.Balari@uab.es
%% and that of ACL 08 by Joakim Nivre and Noah Smith

\documentclass[11pt]{article}
\usepackage{acl2014}
\usepackage{times}
\usepackage{url}
\usepackage{latexsym}

%\setlength\titlebox{5cm}

% You can expand the titlebox if you need extra space
% to show all the authors. Please do not make the titlebox
% smaller than 5cm (the original size); we will check this
% in the camera-ready version and ask you to change it back.


\title{Healthy food: from packaging to consumption}
\date{December 2018}

\author{Matthieu Buot De L'Epine  \\\And
  Valerian Rey  \\\And
Adrien Vandenbroucque \\}

\begin{document}
\maketitle
\begin{abstract}
  \textbf{This document contains a summary of the different steps we took when analyzing the Open Food Facts dataset. We will explain precisely how we first cleaned the dataset and performed many conversions. In a second part, we explain the analysis that we made together with meaningful insights we got.\\
  Finally, we will show the results, and see the conclusions we can draw from them with respect to our problematic.}
\end{abstract}

\section{Introduction}
One of the highest stake of the century is to limit the damages done to natural resources. Understanding where the food is produced, packaged and consumed can show insights about the current issues in the food industry. In some countries, there is a tendency to favor local products, we want to see if this is really the case and see which countries import-export where.

Also, today, people start to buy more and more organic products, because they want to eat healthy and make a positive impact on the environment. At the same time, packaging has become an important matter and there is a need to reduce our plastic waste. Now can we see these new trends in the Open Food Facts dataset? And also, are there any specific characteristics in the packaging of organic products?

Our project is motivated by the idea of understanding these tendencies in the marketing behind the food industry, and get meaningful visualizations of these trends.

\section{Dataset}

\subsection{Open Food Facts}
\subsubsection{Brief description}
Open Food Facts is a food products database made by everyone, for everyone.

\subsubsection{What the dataset contains}
The dataset we used was in CSV format, and list hundreds thousands of products. The information that were useful for our project are the following:

\begin{itemize}
\itemsep0em
    \item Product names
    \item Origin countries
    \item Destination countries
    \item Packaging
    \item Labels
    \item Image URL
    \item Nutrition values
\end{itemize}

\subsubsection{Statistics about the dataset}
The thing about the Open Food Facts dataset is that it is continuously updated, so we cannot give the exact number of products in the database. At the time of the writing of this report, there are around 700'000 products.

\subsubsection{A note about the dataset}
A very important point that need to be made right at the start is that this dataset was created in France, and so many of the products added come from France and Europe. We thus warn the reader that this may introduce significant bias into the analysis, and we therefore ask to be careful not to get hasty conclusions.

\section{Data Cleaning}

The biggest part of our work surely went for the cleaning of the dataset. Indeed, since the dataset is "made by everyone, for everyone", the format of the different columns of the CSV happened to be quite chaotic. For example, many languages are used in the dataset. We explain here the main things we changed in order to make use of most of the data at our disposal.

\subsection{Proportion of missing values}
We were quite surprised to notice that for most of the columns, the proportion of missing values exceeded 50\%. Actually, only 30 over 173 columns had less than 50\% of missing values. So in a first step, we had to carefully chose the columns we wanted to keep in order to perform our analysis.

\subsection{Countries}
After having chosen the columns to keep, the next step was to clean the columns themselves. For the columns containing country names, a lot of work had to be done. Not only contributors wrote in many languages, so we end up with values like "\textit{Royaume-Uni}", "\textit{United Kingdom}" or "\textit{UK}" which all represent one single country. But sometimes, they also indicated the exact region, like "\textit{California}", instead of a country name.\\ 
We designed a way to efficiently translate most of them to English and to the corresponding country with the help of the library \texttt{PyCountry}.

\subsection{Packaging}
For the analysis of the packaging, we also needed to translate the terms to a single language, otherwise when grouping by packaging, we ended up with different groups (one in each language) for one type packaging. For example, for the plastic type of packaging, we could have "\textit{plastico}" (Spanish), "\textit{plastique}" (French), or "\textit{Kunststoff}" (German).\\
In order to clean this, we kept the 10 most frequent used languages, and manually created mappings to translate the main types of packaging (plastic, cardboard, glass or metal).

\subsection{Columns containing lists}
For columns containing information about countries and packaging, it was often the case that there are multiple values in the form of a list, like "\textit{Germany, France}" for countries, or "\textit{plastique, bouteille}" for packaging.\\
Since we needed all countries in those lists for our analysis, we created a function that duplicate lines when there are multiple countries, one for each. Indeed, when grouping by countries, we want that if a product is sold in both Germany and France, then it appears in both groups. The same applies for the packaging.

\subsection{Getting specific labels}

\section{Analysis}

We now turn to the actual analyses we performed on the cleaned dataset. Note that, since it contains many missing values, we split it into multiple smaller datasets so that for each of our analyses, we can use as much data as possible and don't remove any useful information.

\subsection{Analyzing import-export}


\subsection{Analyzing the packaging}


\subsection{The link with organic products}
A big part of our project was to perform the analyses we made just before about import-export and packaging, but specifically on organic products. The methods we used are the same and the results are presented in the next section.
We want now to highlight another big analysis of our project: the extraction of dominant colors of the images of organic products.

In order to perform such a work, we first needed to retrieve the images for the organic products using the images URLs that were available in another file on the Open Food Facts website. While exploring, we saw that the pictures sometimes were taken by a smartphone, and so the background colors would be the main ones. So we decided to first detect the text on the images using the OpenCV library, and then perform the color detection around this area, since it will indeed contain the color of the package and not any incorrect ones.


\section{Results}


\section{Conclusion}


\end{document}
